\chapter{Sistema de resoluci�n de conflictos}
En este cap�tulo se propondr� el marco te�rico y el esquema aplicativo del sistema de resoluci�n de conflictos a utilizar en el marco de MAPC. Es importante destacar que, si bien por limitaciones temporales nuestro equipo �nicamente implement� coordinaciones impl�citas entre las acciones de los agentes durante la competencia del a�o 2011, la idea de implementaci�n de un sistema de coordinaci�n m�s complejo y robusto fue parte de los planes desde el comienzo del trabajo.

\section{Informaci�n necesaria}
Despu�s de estudiar el flujo de informaci�n y el contenido de la base de conocimiento de cada uno de nuestros agentes, sabemos que en cada turno un agente conoce su conjunto completo de creencias, deseos, y tiene al menos una intenci�n persistente y bien determinada, al igual que \textit{al menos} una acci�n a realizar a futuro. Recordemos, adem�s, que gracias al Servidor de Percepciones cada agente tiene una visi�n unificada del mundo y conoce los roles, posiciones y caracter�sticas f�sicas de sus compa�eros de equipo. Toda esta informaci�n puede ser utilizada con el fin de re-analizar concretamente cu�n potencialmente beneficiosa, peligrosa o perjudicial es la acci�n a realizar por el agente, y reconsiderar al respecto.