\chapter{Introducci�n}

\section{Enunciado}

El objetivo de este trabajo es presentar en detalle un esquema de resoluci�n de
conflictos a utilizar en el sistema desarrollado en el marco de MAPC 2011\footnote{Multi-Agent Programming
Contest 2011 - http://www.multiagentcontest.org/}, la representaci�n de la base
de conocimiento utilizada para ello, las decisiones de dise�o e implementaci�n
tomadas, sus fortalezas, debilidades, y posibilidades de trabajo a futuro para
mejorarlo. % mentira sobre multi-agentes y coordinacion, por que resulta copado, etc

A diferencia de los a�os anteriores, esta competencia apunta a lograr un buen
Sistema Multi-Agente puramente distribuido; esto significa que cada agente debe
realizar el proceso de toma de decisiones por su cuenta, en lugar de
contar con una inteligencia centralizada que decida cu�l ser� la acci�n a
realizar para cada uno de los agentes. Este esquema implica que la coordinaci�n a
la hora de la toma de decisiones entre los agentes sea notablemente m�s compleja que
cuando las decisiones se toman de manera centralizada.

A modo de ejemplo introductorio, consideremos un escenario multi-agente cualquiera en el que agentes inteligentes realizan diferentes acciones de manera cooperativa; en el caso de que dos de los agentes realicen una acci�n id�ntica a la vez, existe la posibilidad de que dicha planificaci�n represente un malgasto de tiempo o recursos para los agentes; este problema elemental de coordinaci�n y otros de mucha mayor complejidad pueden ser encontrados en diversos escenarios multi-agente (especialmente cuando la toma de decisiones de cada agente es realizado de manera distribuida), y sus consecuencias en particular en el escenario de la MAPC 2011 motivan el desarrollo de nuestro \textit{sistema de resoluci�n de conflictos}.

El \textit{sistema de resoluci�n de conflictos} funciona de manera local a cada agente,
y tiene como objetivo reconsiderar la acci�n a realizar por dicho agente, una
vez terminado el proceso de decisi�n, analizando las acciones a realizar por el 
resto de los compa�eros de equipo.

\section{Organizaci�n del trabajo}
El trabajo estar� dividido en diferentes cap�tulos para una mejor organizaci�n. Comenzaremos dando un conjunto de definiciones de alto nivel y un marco te�rico (incluyendo un background formal y el contexto de desarrollo) sobre el que trabajaremos durante el resto de la tesis. A continuaci�n, presentaremos en detalle la arquitectura de los agentes utilizados en la competencia (incluyendo la entrada y salida de datos, las diferentes secciones de procesamiento, y la comunicaci�n con otros agentes), e introduciremos las modificaciones necesarias para agregar la resoluci�n de conflictos. Luego describiremos detalladamente dicho sistema, presentando la factibilidad y el marco te�rico para su completa realizaci�n, algunas consideraciones de la implementaci�n en estado beta actual, sus ventajas respecto al resto de los mecanismos de coordinaci�n de la competencia, y algunas posibilidades de trabajo a futuro. Por �ltimo, discutiremos algunas conclusiones globales y alternativas de trabajo para la coordinaci�n en el sistema.

