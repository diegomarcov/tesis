\chapter{Introducci�n}

\section{Enunciado}

El objetivo de este trabajo es presentar en detalle el sistema de resoluci�n de
conflictos utilizado en el marco de MAPC 2011\footnote{Multi-Agent Programming
Contest 2011 - http://www.multiagentcontest.org/}, la representaci�n de la base
de conocimiento utilizada para ello, las decisiones de dise�o e implementaci�n
tomadas, sus fortalezas, debilidades, y posibilidades de trabajo a futuro para
mejorarlo.

% mentira sobre multi-agentes y coordinacion, por que resulta copado, etc

El sistema de resoluci'on de conflictos funciona de manera local a cada agente,
y tiene como objetivo reconsiderar la acci'on a realizar por dicho agente, una
vez terminado el proceso de decisi'on, analizando las potenciales acciones del
resto de los compa�eros de equipo.

El trabajo estar'a dividido en diferentes cap'itulos para una mejor organizaci'on. Comenzaremos dando un conjunto de definiciones de alto nivel y un marco te'orico (incluyendo un background formal y el contexto de desarrollo) sobre el que trabajaremos durante el resto de la tesis. A continuaci'on, presentaremos en detalle la arquitectura de los agentes utilizados en la competencia (incluyendo la entrada y salida de datos, las diferentes secciones de procesamiento, y la comunicacion con otros agentes). Luego describiremos delladamente el mecanismo de resoluci'on de conflictos, presentando la posibilidad y el marco teorico para su completa realizacion, sus ventajas respecto al resto de los mecanismos de coordinaci'on de la competencia, y algunas posibilidades de trabajo a futuro. Por 'ultimo, mostraremos algunas conclusiones globales y alternativas de trabajo para la coordinaci'on en el sistema.


%
%
%\section{Estructura del trabajo}
%
%	\begin{itemize}
%	\item \textbf{Cap�tulo 1}: Introducci�n
%	
%	Contiene la informaci�n b�sica y a alto nivel del contenido del trabajo.
%	
%	\item \textbf{Cap�tulo 2}: Definiciones
%	
%	Contiene las definiciones de los conceptos utilizados a lo largo del trabajo.
%	
%	\item \textbf{Cap�tulo 3}: Something something
%	
%	Contiene texto (!).
%	
%	\item \textbf{Cap�tulo 4}: Something MORE
%	
%	Contiene a�n m�s texto.
%	
%	\item \textbf{Cap�tulo 5}: Conclusiones
%	
%	A partir del an�lisis en los cap�tulos anteriores, delimitamos los resultados del trabajo.
%	
%	\end{itemize}
