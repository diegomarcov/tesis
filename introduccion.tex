\chapter{Introducci�n}

\section{Enunciado}

El objetivo de este trabajo es presentar en detalle un esquema de resoluci�n de
conflictos a utilizar en el sistema desarrollado en el marco de MAPC 2011\footnote{Multi-Agent Programming
Contest 2011 - http://www.multiagentcontest.org/}, la representaci�n de la base
de conocimiento utilizada para ello, las decisiones de dise�o e implementaci�n
tomadas, sus fortalezas, debilidades, y posibilidades de trabajo a futuro para
mejorarlo. % mentira sobre multi-agentes y coordinacion, por que resulta copado, etc

A modo de ejemplo introductorio, consideremos el escenario de la competencia MAPC 2011 (cuya descripci�n puede ser encontrada en el cap�tulo siguiente), y asumamos una zona controlada por agentes de un equipo; si dos de los agentes colocados en nodos consecutivos del mapa decidieran "`expandir"' dicha zona, y cada uno de ellos se moviera hacia un lado opuesto a la vez, la "`zona"' formada por ellos quedar�a destruida puesto que habr�a m�s de un nodo de distancia separ�ndolos. �ste y otros tipos de problemas de coordinaci�n son los que motivan el desarrollo de un \textit{sistema de resoluci�n de conflictos}.

El \textit{sistema de resoluci�n de conflictos} funciona de manera local a cada agente,
y tiene como objetivo reconsiderar la acci�n a realizar por dicho agente, una
vez terminado el proceso de decisi�n, analizando las acciones a realizar por el 
resto de los compa�eros de equipo.

El trabajo estar� dividido en diferentes cap�tulos para una mejor organizaci�n. Comenzaremos dando un conjunto de definiciones de alto nivel y un marco te�rico (incluyendo un background formal y el contexto de desarrollo) sobre el que trabajaremos durante el resto de la tesis. A continuaci�n, presentaremos en detalle la arquitectura de los agentes utilizados en la competencia (incluyendo la entrada y salida de datos, las diferentes secciones de procesamiento, y la comunicacion con otros agentes), e introduciremos las modificaciones necesarias para agregar la resoluci�n de conflictos. Luego describiremos detalladamente dicho sistema, presentando la factibilidad y el marco te�rico para su completa realizaci�n, algunas consideraciones de la implementaci�n en estado beta actual, sus ventajas respecto al resto de los mecanismos de coordinaci�n de la competencia, y algunas posibilidades de trabajo a futuro. Por �ltimo, discutiremos algunas conclusiones globales y alternativas de trabajo para la coordinaci�n en el sistema.

