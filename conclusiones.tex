\chapter{Conclusiones}
% mencion sobre trabajos futuros
% eventual comparacion con mecanismos de coordinacion de otros equipos
% chapear con que nuestro mecanismo de coordinacion es distribuido, y el resto son todos putos
Es importante destacar que, si bien por limitaciones temporales nuestro equipo �nicamente implement� coordinaciones impl�citas entre las acciones de los agentes durante la competencia del a�o 2011, la idea de implementaci�n de un sistema de coordinaci�n m�s complejo y robusto fue parte de los planes desde el comienzo del trabajo.
 
Podemos concluir que el sistema de resoluci�n propuesto resulta en una mejora muy importante en las acciones a tomar por los agentes; si bien en su estado actual puede considerarse m�s cauta de lo necesario, la propuesta abre las puertas a mejoras a�n m�s dram�ticas para la coordinaci�n de los agentes, y cuenta con la enorme ventaja de seguir un esquema puramente distribuido como el solicitado en el marco de la competencia MAPC 2011.

Utilizando por �ltima vez el ejemplo mencionado desde la introducci�n de la tesis, aplicando el sistema de resoluci�n propuesto, s�lo uno de los agentes realizar�a la acci�n excluyente, permitiendo que el resto realice otro tipo de acciones que beneficien al equipo de manera diferentes; en situaciones de conflicto m�s complejas, como una expansi�n descoordinada de una zona que resulte en la p�rdida de la misma, el sistema impedir�a dicha p�rdida, resultando adem�s en que el equipo efectivamente expanda su zona de dominio, pero de una manera lo suficientemente cauta como para que los puntos obtenidos aumenten, y no se corra riesgo de p�rdida de dominio.

El trabajo presentado sirve a modo de fuerte base te�rica si se deseara implementar el sistema para la competencia del a�o 2012; y de efectivizarse adem�s los trabajos a futuro presentados, los agentes podr�an actuar de manera muy cercana a los movimientos �ptimos a realizar durante la competencia (recordando en este caso la subjetividad de las acciones "`�ptimas"', pues no puede considerarse la dimensi�n casi infinita del �rbol de acciones a realizar por el equipo rival).