\chapter{Definiciones}
En este cap�tulo se revisar�n algunas definiciones de conceptos t�cnicos, para posteriormente utilizarlos con tranquilidad en el resto de la presentaci�n.

\section*{Agente}
Un agente es una entidad computacional aut�noma, que puede percibir su entorno a trav�s de sensores, y actuar en dicho entorno utilizando efectores. Usualmente, la informaci�n que un agente percibe de su entorno es s�lo parcial.
\section*{Sistema Multi-Agente}
Es un sistema en el cual muchos agentes interact�an para conseguir alg�n objetivo o realizar alguna tarea.
\section*{Inteligencia Artificial Distribuida}
Es el estudio, construcci�n y aplicaci�n de Sistemas Multi-Agente
\section*{Arquitectura BDI}
El modelo BDI (\textit{Belief-Desire-Intentions}) es tanto una teor�a como una arquitectura de agentes. Un agente BDI posee un conjunto de \textbf{creencias}, un conjunto de \textbf{deseos}, y otro de \textbf{intenciones}. Las intenciones representan un conjunto de alternativas que el agente posee para alcanzar sus metas, y tienen la propiedad de ser \textbf{persistentes}; el agente mantiene todas sus intenciones hasta que logre cumplirlas, o bien se de cuenta de que \textit{no} podr� cumplirlas, o bien, por alg�n otro motivo, las razones que tuvo para adoptar por primera vez a la intenci�n dejaron de ser v�lidas.\\
A grandes rasgos, la arquitectura BDI propone que el agente:
\begin{enumerate}
	\item Perciba los cambios del entorno en el que se desenvuelve.
	\item Revise sus creencias del mundo en base a la percepci�n.
	\item Razone acerca de sus intenciones para reconsiderarlas, de ser necesario.
	\item Seleccione una acci�n a seguir, razonando sobre sus creencias e intenciones.
	\item Ejecute la acci�n seleccionada, y vuelva al primer paso.
\end{enumerate}
